\documentclass{article}
\usepackage[utf8]{inputenc}

\begin{document}
    
\title{Topological analysis of classical and quantum telecommunication networks}
\author{Maria Gragera Garces}

\date{September 2022}

\maketitle

\begin{abstract}
    In this report we compare the resulting fidelity and time taken for end to end communications of different classical and quantum topological communications networks, with the intend of analyzing potential differences between classical and quantum networks on the sole base of topological set up.
    All test have been run in NetSquid, a discrete event quantum simulator designed for quantum networking.
    \end{abstract}

\section{Introduction}

\paragraph{Quantum networks facilitate the transmission of quantum bits, also known as qubits, between quantum processors which can perform quantum logic gates.
The nature of qubits and quantum logic gates is intrinsically different from their classical counterparts. Quantum logic gates are reversible, qubits adhere to a non binary encoding system, and respect quantum mechanic's rules, which modify their behaviour in certain settings when compared to bits.
This hardware difference between both technologies has repercussions in the subsequent networks. As of today, quantum networks highly rely on the entanglement between qubits and node-to-node teleportation of information, which are not available in classical networks.
When working with quantum networks, we can observe the effects of quantum mechanics in settings such as quantum walks \cite{Quantumwalks}. , which showcase a potential behavioural difference between classical and quantum connections within different topological settings.}


\paragraph{Given the novelty of the technology, we don't yet know what the applications for quantum networks will look like in the scale market. This technology might translate to an upgrade of current systems, a dual-format communications format or fully replace classical networks. 
If either of the later scenarios come to be, it will be important to understand what factors affect fidelity and speed in quantum and classical channels, and how these compare against each other. We might end up in scenarios in which classical communication is still preferable to quantum due to an environmental factor of the network such as it's topology.
In this study we will analyze this problem from the perspective of the mathematical topological set up behind our network.}


\section{Experimental set up and topological generators}

\subsection*{NetSquid, a discrete event simulator}
\paragraph{The experiment behind this study was run on Netsquid\cite{Netsquid} , a Network Simulator for Quantum information using Driscrete events.
Four topologies were tested in this study: a line topology, a 2D grid topology, a star topology, and a circle topology.
\begin{figure}
    \centering
    \includegraphics[width=1.1\columnwidth]{d2.png}
    \caption{The  force generated by the controlled expansion and compression of a Hook's law spring. Line indicates linear last square fit with parameters as presented in the main text. Error bars indicate the standard deviation of each individual measurement. \label{d2}} %
    \end{figure}

}

\subsection*{A binary network generator}

\section{Results}

\subsection*{Fidelity}

\subsection*{Time}


\section{Conclusion}

All relevant code is currently open source and available at \href{https://github.com/mgg39/Network-model-comparison}{Network model comparison}

\bibliographystyle{plain}
\bibliography{References_bib.bib}


\end{document}
